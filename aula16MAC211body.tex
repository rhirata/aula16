% Copyright 2007 by Till Tantau
%
% This file may be distributed and/or modified
%
% 1. under the LaTeX Project Public License and/or
% 2. under the GNU Public License.
%
% See the file doc/licenses/LICENSE for more details.

%
% DO NOT USE THIS FILE AS A TEMPLATE FOR YOUR OWN TALKS¡!!
%
% Use a file in the directory solutions instead.
% They are much better suited.
%


\lecture[]{CMake e Controle de versão}{lecture-text}

\subtitle{Conceitos e técnicas}

\date{24/4/2012}

\begin{document}

\begin{frame}
  \maketitle
\end{frame}

\section*{Libraries}

\begin{frame}{Bibliotecas} 
  \begin{itemize}
  \item Esta parte da aula de hoje é baseada no texto:
  \item Program library howto
  \item David A. Wheeler
  \pause 
  \item E no tutorial de YoLinux.com: Static, Shared Dynamic and Loadable Linux Libraries
  \end{itemize}
\end{frame}

\begin{frame}{Bibliotecas} 
  \begin{itemize}
   \item Static: coleção de arquivos de códigos objetos.
   \pause
   \item Shared: também uma coleção de arquivos de códigos objetos, mas preparado
         para ser carregado quando o programa é iniciado.
  \end{itemize}
\end{frame}

\begin{frame}{Bibliotecas estáticas} 
  \begin{itemize}
   \item Bibliotecas estáticas são criadas com o programa \texttt{ar}
  \pause
  \item \texttt{ar rcs libctest.a file1.o file2.o}
  \pause
  \item Uma vez criada, pode-se usá-la usando a opção \texttt{-l} do gcc.
  \pause
  \item \texttt{gcc -Wall -o teste prog.c -L/path/to/library-directory -lctest}
  \pause
  \item \texttt{gcc -Wall -o teste prog.c libctest.a}
  \end{itemize}
\end{frame}


\begin{frame}{Bibliotecas dinâmicas} 
  \begin{itemize}
   \item Bibliotecas dinâmicas são carregadas quando o programa é iniciado.
  \pause
  \item Isso implica que, quando instaladas corretamente, todos os 
        programas que iniciam depois da instalação, automaticamente 
        possam usar a nova biblioteca.
  \pause
  \item As bibliotecas podem ser atualizadas e os programas 
        que queiram usar versões mais antigas e não compatíveis com a 
        versão nova não quebrem.
  \pause
  \item Permite que as bibliotecas, ou funções específicas, possam ser sobrescritas
        quando executando um programa particular. 
  \pause
  \item Fazer tudo isso enquanto os programas estão sendo executados 
        e usando as bibliotecas existentes.
  \end{itemize}
\end{frame}


\begin{frame}{Bibliotecas dinâmicas} 
  \begin{itemize}
   \item Para isso, algumas convenções de nomes são importantes.
  \pause
  \item Toda biblioteca tem um nome especial chamado de "soname". 
  \pause
  \item Ele tem um prefixo "lib", o nome da biblioteca, o sufixo ".so"
  \item seguido por um ponto e o número da versão (que é incrementado quando
        a versão é mudada).
  \pause
  \item Em verdade, um soname qualificado é normalmente um link simbólico para a biblioteca.
  \end{itemize}
\end{frame}


\begin{frame}{Bibliotecas dinâmicas} 
  \begin{itemize}
   \item As bibliotecas dinâmicas podem ser colocadas em qualquer lugar, mas a 
         convenção da GNU é colocá-las em \texttt{/usr/local/lib}
  \pause
  \item O carregador de bibliotecas é o \texttt{/lib/ld-linux.so.X}
  \pause
  \item Por default, uma vez carregado, ele busca as bibliotecas no arquivo de 
        configuração \texttt{/etc/ld.so.conf}
  \pause
  \item Caso queira-se sobrescrever bibliotecas, ou funções, seus nomes devem estar no
        arquivo \texttt{/etc/ld.so.preload}
  \pause
  \item Como procurar nesses diretórios pode ser ineficiente, um programa pré-carrega
        esses nomes: \texttt{ldconfig}. 
  \end{itemize}
\end{frame}


\begin{frame}{Bibliotecas dinâmicas} 
  \begin{itemize}
   \item Há ainda formas de um usuário criar suas próprias bibliotecas e sobrescrever as existentes.
  \pause
  \item Para isso, usam-se as variáeis de ambiente \texttt{LD\_LIBRARY\_PATH}, \texttt{LD\_PRELOAD} 
        e \texttt{LD\_DEBUG}
  \pause
  \item \texttt{LD\_LIBRARY\_PATH} conta para o sistema onde procurar as bibliotecas dinâmicas.
  \pause
  \item \texttt{LD\_PRELOAD} conta para o sistema onde procurar as bibliotecas que irão sobrescrever as existentes.
  \pause
  \item \texttt{LD\_DEBUG} torna o processo verborrático. 
  \end{itemize}
\end{frame}

\begin{frame}{Bibliotecas dinâmicas} 
  \begin{itemize}
   \item Bibliotecas dinâmicas são criadas com o programa \texttt{gcc} 
         e a opção \texttt{-fPIC}, ou \texttt{-fpic}.
  \pause
  \item \texttt{gcc -fPIC -g -c -Wall file1.c}
  \pause
  \item \texttt{gcc -fPIC -g -c -Wall file2.c}
  \pause
  \item \texttt{gcc -shared -fPIC -Wl,-soname,libctest.so.1 -o libctest.so.1.0.1 file1.o file2.o}
  \pause
  \item \texttt{mv libctest.so.1.0.1 /usr/local/lib}
  \pause
  \item \texttt{ldconfig -n}
  \pause
  \item Para usar, basta compilar com as opções -L e -l, como anteriormente.
  \end{itemize}
\end{frame}


%
% \begin{frame}{} 
%   \begin{enumerate}
%   \item
%   \pause
%   \item
%   \pause
%   \item
%   \pause
%   \item
%   \pause
%   \item
%   \end{enumerate}
% \end{frame}

\end{document}






